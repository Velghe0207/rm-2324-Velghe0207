%==============================================================================
% Template research proposal bachelor thesis


\documentclass[english]{hogent-article}

% Invoegen bibliografiebestand
\addbibresource{references.bib}


\studyprogramme{Bachelor of applied information technology}
\course{Research Methods}
\assignmenttype{Paper Research Methods: research proposal}
\academicyear{2023-2024} 

% TODO: Werktitel
\title{Can OS integrated artificial intelligence revolutionize task management for people with ADHD compared to current AI software?}

% TODO: Studentnaam en emailadres invullen
\author{Emile Velghe}
\email{emile.velghe@student.hogent.be}

% TODO (phase 1): Co-author name and email address
% If you write the proposal in collaboration with another student, give their
% name and e-mail address here. If you write the proposal alone, remove these
% lines or comment them out.

% TODO (phase 1): Give the link to your Github-repository here
\projectrepo{https://github.com/hogenttin/rm-2324-Velghe0207}

% Binnen welke specialisatierichting uit 3TI situeert dit onderzoek zich?
% Kies uit deze lijst:
%
% - Mobile \& Enterprise development
% - AI \& Data Engineering
% - Functional \& Business Analysis
% - System \& Network Administrator
% - Mainframe Expert
% - Als het onderzoek niet past binnen een van deze domeinen specifieer je deze
%   zelf
%
\specialisation{AI \& Data Engineering}
\keywords{ADHD, AI, task management, OS}


\begin{document}

\begin{abstract}
  People with ADHD (Attention Deficit Hyperactivity Disorder) often have difficulty organizing their daily obligations and prioritizing tasks because of their lack of focus. To help with this, calendar apps started using AI to automate people’s schedules. While these applications promise many features and customization options, they often require a long and time-consuming setup process to be truly useful. This can sometimes be too overwhelming for people with ADHD, so they end up not using the app after all. Apple's new AI software, however, could potentially solve this problem, as it no longer requires any configuration before it can be used. This study investigates the difficulties people with ADHD experience when using an online calendar app and explores how new artificial intelligence could potentially be the solution to these problems. This happens by having a selected group of people with ADHD use existing apps as their calendar, and then compare their experiences to when they are using newly developed software as their calendar. It is expected for these people to find the new software much more convenient, as it is truly custom designed for people with ADHD. 
\end{abstract}

\tableofcontents

\bigskip

% TODO: EP3
%
% Als je dit voorstel indient in EP3, haal de tekst hieronder uit
% commentaar en pas aan voor jouw situatie.
%
%\paragraph{Opmerking: verbeteringen t.o.v.\ origineel voorstel}
%
% Beschrijf hier kort de verschillen en/of verbeteringen t.o.v. je originele
% voorstel.

% TODO: Bachelorproef
% 
% Neem je dit jaar ook de bachelorproef op? Haal dan de tekst hieronder
% uit commentaar en pas aan voor jouw situatie.
%
%\paragraph{Opmerking}
%
% Ik neem dit jaar ook de bachelorproef op. De inhoud van dit onderzoeksvoorstel dient ook als het onderwerpvoor mijn bachelorproef. Mijn promotor is (Mr./Mevr.) X.\ Familienaam.
%
% Beschrijf de eventuele verschillen en/of verbeteringen in dit document t.o.v.\ jouw onderzoeksvoorstel dat je ingediend hebt voor de bachelorproef.

\section{Inleiding}%
\label{sec:inleiding}

% TODO: (fase 1 - onderzoeksvraag formuleren)

One of the main difficulties people with ADHD experience is task and time management, as they often lose track of time very quickly, and then forget important things as a result. Recently, developers started implementing artificial intelligence and machine learning into apps to automate task management. This way, the app could provide the user with smart suggestions. To make the experience even more personal,  the software usually requires some sort of setup process or survey in which the user provides them with personal information. Also, although the apps are called “smart” or “intelligent”, the user still has to manually enter its appointments or reminders. People with ADHD often don't have the patience or mental motivation to go through this process and so, they end up abandoning the app. However, with the help of Apple's new AI software, this could possibly be the solution for people with ADHD to help them maintain their schedule, as it has access to all their personal data and can take immediate action across apps. This study describes why modern AI powered calendar apps are not optimized for people with ADHD and why new AI could possibly change that. The literature review will first investigate and describe the different kinds of AI software used today and what their limitations are. This software will then be compared to new AI technology to determine the main differences in functionality. Lastly, the study will determine the potential this new software has in assisting people with ADHD with time management and so, how this can add huge value to their lives.

\section{Literatuurstudie}%
\label{sec:literatuurstudie}

% TODO: (fase 3, 4 - literatuurstudie)

People often think having ADHD means getting easily distracted and acting impulsive, but in fact, ADHD means way more than just that. It is often the cause of many more things people with ADHD struggle with in their day to day life. This can sometimes lead to a negative self-esteem as they feel like they can’t get a hold on their lives and underperform when working on something

Major theories of ADHD have emphasized core deficits in the cognitive processes associated with attention and executive function (EF), which includes processes such as behavioral inhibition, working memory, and organization and planning skills \autocite{Molitor2017}.

Normally, people would just start using any kind of calendar either online or on paper when they have to schedule appointments or be reminded about important tasks. People with ADHD however, find it just as difficult to build and follow a schedule because that is just the way their impulsive brain works. They can never predict what they will be doing in the future. In the real early stages of software development, apps were designed specifically to help people in general with planning their days. It was basically just a visual of your calendar as you would normally have on paper, but now it was on your phone. These apps changed nothing to the general activity of planning. This is where artificial intelligence could normally offer some help, as it is designed to think like real humans. 

AI has the potential to improve treatment for individuals with ADHD. For example, AI-powered virtual assistant or chatbot could provide round-the-clock support, guidance, and scheduling to improve attention and executive functioning skills in managing ADHD. A similar technology has already been adapted for other mental health conditions, such as anxiety, depression, and stress. \autocite{Rahman2023}.

Artificial intelligence has improved significantly over the last five to ten years and has been implemented into a lot of different applications. One of these is the use of AI in day to day scheduling and task management apps. Popular apps such as: Todoist, Trello, and Microsoft To Do have already shown to be very helpful with this. These apps can assist people with ADHD by automating and simplifying the process of task organization and time management. They use AI to learn about a user’s habits and preferences and constantly adapt to this to provide even more accurate support. 

According to the research by Cho and Kim, LLMs effectively engaged patients empathetically and adaptively. However, it also identified limitations in personal interactions and understanding emotions to the depth that a human therapist would do \autocite{Berrezueta-Guzman2024}.

One of the biggest downsides about current models is the need for users to share personal information with their AI system to personalize responses to their own situation. In the early stages of AI development, the level of “smart” assistance AI powered applications offered, when used for the first time, was generally the same for every user, as they were trained by the same data sets. These datasets provide the AI software with one general image of how the average person schedules their day. This means that the app could now, for example, estimate how long it would take for a simple task to complete and change the schedule accordingly.

To change this, developers need information about real individuals and how they behave in a day to day life, but the real question is where to get this information. 
The only real personal data these apps, and other software applications, have access to, is basic personal information such as: age, sex, country of birth, things you like, and often your location as well. This is the only information separating assistance for one user to another. By using the app for an extended amount of time, it learns more about the user and its habits to provide more personal suggestions. However, the app still doesn't have access, and also won’t have access in the future,  to real personal data such as: texts, photos, notes or calls.

Thanks to AI, all kinds of personal data can be used to analyse, forecast and influence human behaviour, an opportunity that transforms such data, and the outcomes of their processing, into valuable commodities. In particular, AI enables automated decision-making even in domains that require complex choices, based on multiple factors and non-predefined criteria \autocite{Sartor2020}.

Therefore, apps that use AI, when just downloaded, try to collect as much information about its user as possible to compete with other apps by offering the best personal user experience. This can be done in multiple ways such as filling out a question form when setting up the app, or giving the app permission to different kinds of data across your device. This information is needed if the app wants to make good on its promises and give a good first impression. For many people, especially those with ADHD, the first impression when using any kind of app for the first couple of days is very important, as this usually determines whether or not they will be using it in the long term. This explains why paid applications usually come with a 7-day free trial. When downloading the app called ‘Motion’ for example, it first asks you to connect your current calendar. It then asks the user to enter some of their own tasks they have to complete and recommends entering at least five, so the app’s AI can respond more accurately. When you complete this setup, you are greeted with a pretty minimal interface that looks basically like any other scheduling interface. Next, a step by step user guide is offered and when you are finally ready to utilize the app, a pop-up asks you to pay over €300 yearly if you want to continue using it

This whole setup process and new interface the user has to get used to could potentially lead to people with ADHD to not use the app, as this takes a lot of mental effort. If a user manages to get through the first days or weeks while using the app as their personal assistance, they will notice more personal suggestions for their situation. This is the result of the AI tracking their behaviour across multiple parameters. It tracks for example your location, sleep schedule, notifications, screen time, content you watched or liked, and combines this data all together to get an idea of your personality. However, knowing some app is tracking every move you make, people started wondering what other data is being monitored and what happens to it once the software has processed it. Given the fact that nowadays, almost every AI software is cloud-based, this essentially means that your personal data is being sent across the internet to various servers and databases. This ultimately raises a lot of questions about privacy among users.

“While AI-powered apps are used to analyze individual’s daily activities and movement patterns to detect signs of ADHD and keep them organized, there is a potential concern about individual’s privacy and identification. These apps may collect sensitive personal information, such as location data, medical history, daily conversations, personal preferences, and behaviour. This information may be misused or shared without a proper consent \autocite{Rahman2023}.

The software generally known today as artificial intelligence is mostly cloud-based. Online applications such as ChatGPT generate their responses on external servers and communicate with the user over the internet. Application based software normally uses more of your own device's computing power but still communicates with servers and databases over a network. Both are incapable of providing their assistance without internet traffic, resulting in a permanent risk of your data getting leaked. Until recently, there was no other option available. With Apple's recent WWDC event, which announces their new products and features for the upcoming year, however, they might have just found the solution to this problem. Their new software, called Apple AI, fixes not only the demand for personalisation, but also ensures optimal data privacy. The way Apple is able to achieve this, is by integrating artificial intelligence software into a mobile device's operating system,  iOS or MacOS in this case. This allows the software to have direct access to all of your personal data, providing the most custom tailored assistance, while ensuring complete privacy.

“It has to be powerful enough to help with the things that matter most to you. It has to be intuitive and easy to use. It has to be deeply integrated into your product experiences. Most importantly, it has to understand you and be grounded in your personal context like your routine, your relationships, your communications and more. And of course, it has to be built with privacy from the ground up. Together, this goes beyond artificial intelligence. It's personal intelligence” \autocite{Cook2024}.

On paper, this new software could resolve almost every problem encountered today with modern AI. However, because Apple Intelligence was only announced recently and is currently only available in beta versions, the value or use cases for people with ADHD have not been tested yet. If Apple does live up to its promises, this software could become a massive improvement for the day to day lives of people with ADHD, when compared to current existing software. 


% Refereren naar de literatuur kan met:
% \autocite{BIBTEXKEY} => (Auteur, jaartal): voor een referentie tussen
% haakjes, waar de auteursnaam GEEN onderdeel is van een zin.
% \textcite{BIBTEXKEY} => Auteur (jaartal): voor een narratieve referentie,
% waar de naam van de auteur effectief een onderdeel is van de zin.

\section{Methodologie}%
\label{sec:methodologie}

% TODO: (fase 5 - methodologie)

Because this software is so different and new, we first learn how to use it and how to formulate the best prompts to get the right answer. Then, multiple different use cases are selected to ensure a clear comparison between this new software and current software when executing ADHD assisting tasks. The use cases are then translated into different categories or criteria. 

After that, the real study starts by interviewing groups of all different kinds of people with ADHD, such as students or employed adults. They will be asked about their use of AI in a day to day life. This way, we can compare the different groups of age, sex, interests, profession in the use of AI applications to assist them with the struggles they experience because of their ADHD. They are then asked to download the most recommended app in every category and use this as their main source of assistance. 

After that, we'll ask them to switch to the new Apple Intelligence and now use this as their new personal assistance. In both cases, people are given a detailed explanation about how they can get the most use out of the software. 

In the end, they are asked about their findings and the pros and cons they experienced. They will eventually have to rate both AI models for every category determined at the start.This should normally translate to an obvious answer to our research question.


% Snippet voor een afbeelding dat je bv kan gebruiken voor een Gantt-diagram.
%
% We gebruiken hier de figure*-omgeving zodat de figuur over beide kolommen
% gespreid wordt voor betere leesbaarheid. Probeer de positionering van
% figuren niet te manipuleren (met bv [ht!]), maar zorg altijd voor een
% zinvol bijschrift en label, en refereer er naar in de tekst.
%
% Bij afbeeldingen die je overneemt, sluit je het bijschrift af met een
% bronvermelding (commando \autocite).
%
% \begin{figure*}
%   \centering
%   \includegraphics[width=\textwidth]{example-image-16x9}
%   \caption{\label{fig:gantt}Gantt diagram met de verschillende fasen en milestones van het onderzoek.}
% \end{figure*}

\section{Verwachte resultaten}%
\label{sec:verwachte-resultaten}

% TODO: (fase 6 - afwerking)

The expectations are for the new Apple Intelligence come out as the ultimate winner and for it to be extremely helpful for people with ADHD. This is because it takes the advantages and use cases from existing software and even improves them while also completely eliminating the current limitations of this sofware.


section{Discussie, verwachte conclusie}%
\label{sec:discussie-conclusie}


%------------------------------------------------------------------------------
% Referentielijst
%------------------------------------------------------------------------------
% TODO: (fase 4) de gerefereerde werken moeten in BibTeX-bestand
% bibliografie.bib voorkomen. Gebruik JabRef om je bibliografie bij te
% houden.

\printbibliography[heading=bibintoc]


\end{document}